% ou should aim to produce around 5 pages of synopsis of the existing work that forms the context for your project. You should give this to your supervisor by the end of week 5, and discuss it with them at your next meeting, jointly filling in and submitting this form.

% It is suggested that you use LaTeX:

% Link: LaTeX instructions and document class files

% Note:

% Although typically published papers will be a substantial part of the background (see more below) this may differ for different projects. For example, your task may instead require you to get a strong understanding of an existing code-base, or critical software tools, or of an application domain. You will still need to demonstrate your knowledge and critical evaluation through writing a good synopsis of the relevant exisiting material, to set your work in context, and justify your approach.
% A good background chapter is not just a summary of all the things you have read. Rather, it should be selective and well-organised, to tell the reader just what they need to know to understand the work that follows.
% It should be clear to the reader (and to you!) why you are including any information that you present. A lengthy, rambling and 'padded' background chapter is a common reason for lower marks on these criteria; or a chapter that omits critical material.
% Probably the best guide to the expected content is to look at published papers in the same domain as your project, and how they put their new work in the context of previous work.
% Note for MInf part 2: your MInf part 1 dissertation forms part of the background literature, and you can refer to it as such. This does not mean you should assume the marker has read it. Rather, you should summarise it (and other relevant material) sufficiently clearly that the reader has the context for understanding your current work. Looking at a published paper that similarly builds directly on previous work from the same authors should be a good guide.
% RESEARCHING THE BACKGROUND
% Your project supervisor should have pointed you towards some key sources for your project, but is expected (certainly to achieve an excellent grade) that you will put some significant effort into going beyond those sources.

% There are many literature search tools you can use, but using Google Scholar as an example, here are some tips, starting from a key reference:

% 'Cited by' will show you all the subsequent papers that refer to your source paper. So if anyone has followed up the work already, you should find it here.
% 'Related articles' show you all the papers that have overlapping references to your source paper. The top hits are very likely to be addressing similar problems.
% The source paper itself will cite other papers, and you should be able to tell from the paper which of these might be crucial (e.g. another paper that was the first presentation of an important method that the current paper uses) and choose which you should read first-hand.
% If you have some key terms, try searching for those terms plus 'review' or 'survey', and limiting the search to recent dates. This can be a good way to find if there is a recent published overview of the topic. Such a paper is also likely to contain references to other useful papers.

\chapter{Background}
